%\begin{document}
\chapter{PC/104运行分析}
\thispagestyle{empty}
%==========================================================================
\section{PC/104简介}
PC/104(pc104)是ISA(IEEE-996)标准的延伸。1992年PC/104作为基本文件被采纳,叫做IEEE-P996.1兼容PC嵌入式模块标准。PC/104是一种专门为嵌入式控制而定义的工业控制总线。IEEE-P996是ISA工业总线规范,IEEE协会将它定义IEEE-P996.1,PC/104实质上就是一种紧凑型的IEEE-P996,其信号定义和PC/AT基本一致,但电气和机械规范却完全不同,是一种优化的、小型、堆栈式结构的嵌入式控制系统。其小型化的尺寸($90\times 96$mm),极低的功耗(典型模块为1-2瓦)和堆栈的总线形式(决定了其高可靠性),受到了众多从事嵌入式产品生产厂商的欢迎,在嵌入式系统中领域逐渐流行开来。

\textbf{\xiaosan PC/104的优点}
\begin{itemize}
  \item\textbf{\xiaosi 大尺寸}\\
    PC/104的板卡标准尺寸为$90\times 96$mm(比一本新华字典还要小很多,而传统桌面PC系统的板卡尺寸为$315\times 122$mm),这样小的尺寸使得PC/104、PC/104+和PCI-104模块板成为了嵌入式系统应用的理想产品。
  \item\textbf{\xiaosi 开放的高可靠性的工业规范}\\ PC/104、PC/104+和PCI-104产品在电气特性和机械特性上可靠性极高,功耗低,产生热量少。板卡与板卡之间通过自堆栈进行可靠的连接,抗震能力强。全世界有超过200家公司使用这些开放的规范来生产和销售各种PC/104模块板。
  \item\textbf{\xiaosi 模块可自用扩展}\\ PC/104模块具有灵活的可扩展性。它允许工程师互换及匹配各种功能卡,可随系统的需求而升级CPU的性能。增加系统的功能和性能只需通过改变相应的模块即可实现。
  \item\textbf{\xiaosi 低功耗}\\ 4mA的总线驱动电流,即可使模块正常工作,低功耗有利于减少元件数量。各种插卡广泛采用VLSI芯片、低功耗的ASIC芯片、门阵列等,其存储采用大容量固态盘(SSD)。
  \item\textbf{\xiaosi 堆栈式连接}\\ 这种结构取消了主板和插槽,可以将所有的PC/104模块板利用板上的叠装总线插座连接起来。有效减小整个系统所占的空间。PC104板的叠装总线插座是针脚插接方式,理论上可以无限扩插N多扩展卡,但要看他的承受能里。
\end{itemize}
%==========================================================================
\section{卷积码序贯译码在PC/104上运行}
在PC/104上,测试$POLY1=0xA5048D,POLY2=0xDAFB73$约束长度为$K=24$的卷积码的运行效率与误码率,下表就是测试的一些数据,其中测试的数据长度为$L=1000000$比特,既是125000字节,又根据第三章编码中,终结码介绍可知,加上24比特,这样可以实现结尾的误码率不受结尾码字的影响。测试环境为BPSK调制,AWGN信道。表\ref{tab:5.1}就是运行得到的数据。
\begin{table}[htpb]
  \centering
   \caption{约束长度$K=24$的卷积码测试数据表}
  \label{tab:5.1}
  \wuhao
  \begin{tabular}{|c||c|c|c|c|c|c|}
    \hline
    $E_b/N_0$&2.0&2.25&2.5&2.75&3.0&3.25\\
    \hline
    比特错误率&$1.001\times 10^{-1}$&$8.93\times 10^{-2}$&$6.24\times
    10^{-2}$&$2.65\times 10^{-2}$&$9.5\times 10^{-3}$&$5.8\times 10^{-3}$\\
    \hline
    \hline
    $E_b/N_0$&3.50&3.75&4.0&4.25&4.5&4.75\\
    \hline
    比特错误率&$1.7\times 10^{-3}$&$7.8\times
    10^{-4}$&$3.12\times 10^{-4}$&$1.15\times 10^{-4}$&$1.0\times
    10^{-5}$&$1.01\times 10^{-6}$\\
    \hline
  \end{tabular}
\end{table}
从表\ref{tab:5.1},结合参考书上给出的实例可以看出,误码率还是符合理论要求的。因此,现在要测试的就是运行的时间,译码运行时间的大小,决定译码时候缓冲区有无以及大小。
表\ref{tab:5.2}将给出在各个$E_b/N_0$下,译码器运行的时间,其中码长为1000024比特。
\begin{table}[htpb]
  \caption{约束长度$K=24$的卷积码运行时间表}
  \label{tab:5.2}
  \centering
  \begin{tabular}{|c||C{1.8cm}|C{1.8cm}|C{1.8cm}|C{1.8cm}|C{1.8cm}|C{1.8cm}|}
    \hline
    $E_b/N_0$&2.0&2.25&2.5&2.75&3.0&3.25\\
    \hline
    运行时间(s)&59.16&50.87&46.39&44.49&40.34&39.50\\
    \hline
    \hline
    $E_b/N_0$&3.50&3.75&4.0&4.25&4.5&4.75\\
    \hline
    运行时间(s)&39.99&35.45&33.57&32.90&30.00&28.5\\
    \hline
  \end{tabular}
 
\end{table}
从图中可以看出,序贯译码,在$E_b/N_0$很低的时候,运行时间需要很长,这是因为,低$E_b/N_0$情况下,序贯译码需要大量的运算,因此回溯次数会增加。而$E_b/N_0$越高相对需要的时间越少。

从总体考虑,缓冲区还是需要的,尤其是传输码率和数据量很大的时候,至于缓冲区大小的确定这要根据数据量的多少,码字传输速率,$E_b/N_0$等情况综合考虑。
%==========================================================================
\section{RS码BM算法译码在PC/104上运行}
在PC/104上,测试$RS(15,9,7)$的运行效率与误码率,下表就是测试的一些数据,错误的帧数为200,且每帧36比特,高于此帧数的时候就停止输入数据,停止编译码。测试环境为BPSK调制,AWGN信道。
\begin{longtable}{|c||c|c|c|c|c|c|}
\caption{$RS(15,9,7)$测试数据表}
\label{tab:5.3}\\ 

\endfirsthead

\multicolumn{7}{c}{续表~\thetable\hskip1em $RS(15,9,7)$测试数据表}\\


\endhead

\hline
\multicolumn{7}{r}{续下页}
\endfoot
\endlastfoot
\hline
    $E_b/N_0$&0.5&0.75&1&1.25&1.5&1.75\\
\hline
\hline
    比特错误率&$1.3\times 10^{-1}$&$1.2\times 10^{-1}$&$1.0\times
    10^{-1}$&$1.0\times 10^{-1}$&$9.8\times 10^{-2}$&$8.3\times
    10^{-2}$\\
\hline
    字节错误率&$6.4\times 10^{-1}$&$6.2\times 10^{-1}$&$6.0\times
    10^{-1}$&$5.8\times 10^{-1}$&$5.2\times 10^{-1}$&$4.7\times
    10^{-1}$\\
\hline
    帧错误率&$9.4\times 10^{-1}$&$9.0\times 10^{-1}$&$8.9\times
    10^{-1}$&$8.7\times 10^{-1}$&$8.5\times 10^{-1}$&$7.6\times
    10^{-1}$\\
\hline
\multicolumn{7}{c}{\vspace{2pt}}\\
\hline
 $E_b/N_0$&2&2.25&2.5&2.75&3&3.25\\
 \hline
 \hline
 比特错误率&$8.3\times 10^{-2}$&$6.0\times 10^{-2}$&$6.1\times 10^{-2}$&$4.9\times 10^{-2}$&$4.4\times
    10^{-2}$&$3.5\times 10^{-2}$\\
\hline
 字节错误率&$4.6\times 10^{-1}$&$3.7\times 10^{-1}$&$3.5\times 10^{-1}$&$3.0\times 10^{-1}$&$2.7\times
    10^{-1}$&$2.2\times 10^{-1}$\\
\hline
帧错误率&$7.5\times 10^{-1}$&$6.4\times 10^{-1}$&$5.9\times 10^{-1}$&$5.2\times 10^{-1}$&$4.7\times
    10^{-1}$&$3.9\times 10^{-1}$\\
\hline
\multicolumn{7}{c}{\vspace{2pt}}\\
\hline
$E_b/N_0$&3.5&3.75&4&4.25&4.5&4.75\\
\hline
\hline
比特错误率&$2.9\times 10^{-2}$&$2.2\times
    10^{-2}$&$1.6\times 10^{-2}$&$1.3\times 10^{-2}$&$9.3\times 10^{-3}$&$7.9\times 10^{-3}$\\
\hline
字节错误率&$1.8\times 10^{-1}$&$1.3\times
    10^{-1}$&$1.1\times 10^{-1}$&$8.6\times 10^{-2}$&$6.0\times 10^{-2}$&$5.2\times 10^{-2}$\\
\hline
帧错误率&$3.2\times 10^{-1}$&$2.4\times
    10^{-1}$&$2.0\times 10^{-1}$&$1.6\times 10^{-1}$&$1.1\times 10^{-1}$&$9.9\times 10^{-2}$\\
\hline
\multicolumn{7}{c}{\vspace{2pt}}\\
\hline
$E_b/N_0$&5&5.25&5.5&5.75&6&6.25\\
\hline
\hline
比特错误率&$4.2\times
    10^{-3}$&$3.1\times 10^{-3}$&$1.8\times 10^{-3}$&$1.1\times
    10^{-3}$&$5.8\times 10^{-4}$&$3.9\times 10^{-4}$\\
\hline
字节错误率&$2.9\times
    10^{-2}$&$1.9\times 10^{-2}$&$1.2\times 10^{-2}$&$7.3\times
    10^{-3}$&$3.9\times 10^{-3}$&$2.6\times 10^{-3}$\\
\hline
帧错误率&$5.5\times
    10^{-2}$&$3.7\times 10^{-2}$&$2.3\times 10^{-2}$&$1.4\times
    10^{-2}$&$7.5\times 10^{-3}$&$5.0\times 10^{-3}$\\
\hline
\multicolumn{7}{c}{\vspace{2pt}}\\
\hline
$E_b/N_0$&6.5&6.75&7&7.25&7.5&7.75\\
\hline
\hline
 比特错误率&$1.9\times 10^{-4}$&$1.2\times 10^{-4}$&$4.6\times
    10^{-5}$&$2.9\times 10^{-5}$&$1.1\times 10^{-5}$&$4.1\times
    10^{-6}$\\
\hline
 字节错误率&$1.3\times 10^{-3}$&$8.1\times 10^{-4}$&$3.1\times
    10^{-4}$&$1.9\times 10^{-4}$&$6.8\times 10^{-5}$&$2.8\times
    10^{-5}$\\
\hline
 帧错误率&$2.4\times 10^{-3}$&$1.6\times 10^{-3}$&$6.1\times
    10^{-4}$&$3.6\times 10^{-4}$&$1.4\times 10^{-4}$&$5.7\times
    10^{-5}$\\
\hline
\end{longtable}

表\ref{tab:5.3}测试了$E_b/N_0$从
2到7.75步进为0.25的比特错误率,字节错误率和帧错误率,表格很细致的反映了RS误码率的走势和未经过信道编码的BPSK调制误码率的比较,从而可以看出,RS还是很好的降低了误码率,满足了我们的要求。


\begin{longtable}{|c||c|c|c|c|c|c|}
\caption{$RS(15,9,7)$运行时间表}
\label{tab:5.4}\\ 

\endfirsthead

\multicolumn{7}{c}{\wuhao\kaiti{续表~\thetable\hskip1em $RS(15,9,7)$运行时间表}}\\


\endhead

\hline
\multicolumn{7}{r}{\wuhao\kaiti{续下页}}
\endfoot
\endlastfoot
\hline

\hline
    $E_b/N_0$&0.5&0.75&1&1.25&1.5&1.75\\
\hline
\hline
运行时间(s)&$1.0\times 10^{-2}$&$1.0\times 10^{-2}$&$1.0\times
    10^{-2}$&$1.0\times 10^{-2}$&$1.0\times 10^{-2}$&$2.0\times
    10^{-2}$\\
\hline
 总比特数&$7.7\times 10^3$&$8.0\times 10^3$&$8.1\times
    10^3$&$8.4\times 10^3$&$8.5\times 10^3$&$9.6\times 10^3$\\
\hline
\multicolumn{7}{c}{\vspace{2pt}}\\
\hline
 $E_b/N_0$&2&2.25&2.5&2.75&3&3.25\\
 \hline
 \hline
 运行时间(s)&$1.0\times 10^{-2}$&$2.0\times 10^{-2}$&$2.0\times 10^{-2}$&$1.0\times 10^{-2}$&$3.0\times
    10^{-2}$&$2.0\times 10^{-2}$\\
\hline
总比特数&$9.7\times 10^3$&$1.1\times 10^4$&$1.2\times 10^4$&$1.4\times 10^4$&$1.5\times
    10^4$&$1.9\times 10^4$\\
\hline
\multicolumn{7}{c}{\vspace{2pt}}\\
\hline
$E_b/N_0$&3.5&3.75&4&4.25&4.5&4.75\\
\hline
\hline
运行时间(s)&$4.0\times 10^{-2}$&$3.0\times
    10^{-2}$&$7.0\times 10^{-2}$&$5.0\times 10^{-2}$&$9.0\times 10^{-2}$&$1.2\times
    10^{-1}$\\
\hline
总比特数&$2.2\times 10^4$&$3.0\times 10^4$&$3.6\times
    10^4$&$4.4\times 10^4$&$6.5\times 10^4$&$7.3\times 10^4$\\
\hline
\multicolumn{7}{c}{\vspace{2pt}}\\
\hline
$E_b/N_0$&5&5.25&5.5&5.75&6&6.25\\
\hline
\hline
 运行时间(s)&$1.7\times 10^{-1}$&$2.8\times 10^{-1}$&$4.2\times 10^{-1}$&$6.7\times 10^{-1}$&$1.2\times 10^0$&$2.1\times 10^0$\\
 总比特数&$1.3\times
    10^5$&$1.9\times 10^5$&$3.1\times 10^5$&$5.1\times 10^5$&$9.6\times 10^5$&$1.4\times
    10^6$\\
\hline
\hline
\multicolumn{7}{c}{\vspace{2pt}}\\
\hline
$E_b/N_0$&6.5&6.75&7&7.25&7.5&7.75\\
\hline
\hline
 运行时间(s)&$3.4\times 10^{0}$&$7.4\times 10^{0}$&$1.4\times
    10^{1}$&$3.4\times 10^{1}$&$7.1\times 10^{1}$&$1.7\times
    10^{2}$\\
\hline
总比特数&$2.8\times 10^6$&$4.5\times 10^6$&$1.2\times
    10^7$&$2.0\times 10^7$&$5.3\times 10^7$&$1.3\times 10^8$\\
\hline
\end{longtable}

从表\ref{tab:5.4}中可以看出,随着$E_b/N_0$的升高,运行时间一直在增加,这里有个误解,事实上并非如此,由于随着$E_b/N_0$的升高,误码率是下降的,为了精确度的原因,必然需要越多的数据进行测试,才能达到误码率精度的要求,因此随着$E_b/N_0$的升高,测试数据也增多,相应的运行时间也是增多的。

通过表中运行时间和测试数据量的关系可以看出,运行时间和$E_b/N_0$并无关系,这也是理论上支持的。而且从运行时间上看,译码算法还是很高效的,在加上本设计的目的和水声通信的要求,RS编译码只用于少量数据的传输,因此缓冲区可以省掉。
%==========================================================================
\section{本章小结}
本小节首先介绍了PC/104,并说明为什么选择PC/104作为我们测试译码算法的平台,其次在给出了序贯译码算法和BM译码算法在PC/104上运行的结果,包括测试误码率数据和运行时间,以及讨论关于缓冲区的选取和大小的选择。
%==========================================================================
%%========================================================================
% empty page for two-page print
\ifthenelse{\equal{\ioaside}{T}}{%
  \newpage\mbox{}%
  \thispagestyle{empty}}{}
%%========================================================================
%\end{document}
