%\begin{document}
\chapter{水声通信系统Turbo均衡算法湖试数据分析}
\thispagestyle{empty}
%==========================================================================
\section{研究工作总结}
由于人们对海洋的关注越来越多,人类在海洋中的活动越来越频繁,因此作为海洋技术基础的水声通信,必然也是人们关注的焦点,而信道编码作为水声通信系统可靠性的保障,也必然研究广泛。

本文分别从编码理论、卷积码编译码已经RS码编译码三个方面介绍水声通信的信道编码。
\begin{enumerate}
  \item
    编码代数理论,作为信道编码的基础,有必要认真的学习一番,这是我们理解和应用各种信道编码的前提,因此在最初,就介绍了信道编码的相关代数基础。
  \item
    卷积码的编译码,从原理和实现两个方面介绍了整个编译码过程,着重讨论了序贯译码方式的费诺算法,并仿真和分析性能。
  \item
    RS码的编译码,也是从原理和实现两个方面介绍了整个编译码过程,着重讨论了BM译码算法,并仿真和分析性能。
\end{enumerate}
%==========================================================================
\section{下一步工作}
\subsection{费诺算法的进一步改进}
在上述章节中提到的费诺算法,我们知道,其运算量很大,尤其是在信噪比很低的时候,因此会有很大的延时,这是系统不能容忍的,而与之对应的另外一种序贯译码算法——多堆栈算法,却能避免大量的运算,只是存储空间的浪费比费诺算法严重的多。

结合这两种算法,使其具有多堆栈算法的效率,又有费诺算法的节省存储空间。
\subsection{Turbo码等其他码字的研究}
由于水声通信和编码技术的日益发展,其他一些新的码字也渐渐地显示出自己的优势,而Turbo无疑是其中的佼佼者,因此,Turbo码也是下一步工作的重点。

最后还有级联码,因为已经做好了卷积码和RS码的编译码,因此把RS码作为外码,卷积码作为内码的级联码的仿真也是一个重点。
%%========================================================================
% empty page for two-page print
\ifthenelse{\equal{\ioaside}{T}}{%
  \newpage\mbox{}%
  \thispagestyle{empty}}{}
%%========================================================================
%\end{document}
