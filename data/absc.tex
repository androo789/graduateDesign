
%% 中文摘要
\chapter*{\centerline{摘\quad 要}}
\chaptermark{摘\quad 要}
\addcontentsline{toc}{chapter}{\heiti\xiaosi 摘\quad 要} 
\pagenumbering{Roman}
%\pagestyle{plain}
\thispagestyle{plainabstractc}
\vspace{1em}
由于声波可以在水声传播很远的距离,水声通信时水下无线信息传输的主要手段。随着人类在海洋中的活动越来越频繁,需要在水下传世信息的应用需求也越来越多,水声通信技术的应用月越来越广泛。在现在和将来的科学考察、海洋工程建设、海底矿产资源调查与开发以及军事领域等方面,水声通信技术都有着广泛的应用。这些应用也对水声通信的速率和可靠性提出了更高的要求。

在现代水声通信系统中,水声信道面临的主要任务就是客服多径效应产生的码间干扰。而克服多径效应,减少码间干扰的主要手段包括空间分集技术和均衡技术。

本文针对水声信道高速数据传输中的码间干扰抑制问题,研究了Turbo均衡技术以及与其相匹配的软迭代信道估计算法。基于先验信息MMSE准则的线性Turbo均衡技术利用译码器反馈的软信息,以最小化均方误差(MMSE)为准则,估计出发送符号的估计值并通过映射得到关于发送符号的软信息。该软信息作为SISO译码器的输入,通过译码器输出关于发送符号的软信息并反馈给均衡器,从而形成回路。通过迭代次数的增加,均衡器的性能会越来越好。

由于基于先验信息MMSE准则的线性Turbo均衡算法需要信道的信息,为了提高均衡效率,本文针对水声信道的特点,提出一种软迭代信道估计算法,该算法利用软的FOLMS算法估计横向滤波系数以及二阶锁相环估计相位,并通过迭代来提高估计性能。

通过理论分析和计算机仿真,研究了线性Turbo均衡和软迭代信道估计算法对水声信道均衡的性能。此外,还将本文提出的算法进行湖试数据验证,证明了所研究的基于先验信息MMSE准则的线性Turbo均衡算法具有很好的实用价值。

本文的主要研究工作如下:
\begin{enumerate}
  \item
      提出基于先验信息MMSE准则的线性Turbo均衡算法与T-TCM相结合,并应用于海试、湖试数据处理中。该算法在减少系统运算复杂度的同时,实现均衡器性能的提升。
  \item
      提出一种适用于水声Turbo均衡的软迭代信道估计算法,该算法利用软的FOLMS算法估计横向滤波系数及二阶锁相环估计相位,并通过迭代提高估计性能,从仿真分析以及海试数据处理可以看出,其性能远远高于硬迭代信道估计算法和不带相位估计器的软迭代信道估计算法。
\end{enumerate}

\vspace{1em}

\noindent\textbf{关键词}

\thispagestyle{plainabstractc}
水声相干通信\quad 水声信道\quad
快速自最优LMS\quad 自适应均衡\quad Turbo均衡\quad 二阶锁相环\quad
比特交织\quad 时间反转

