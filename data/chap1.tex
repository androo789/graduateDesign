%\documentclass[twoside,numberorder]{buptthesis}
%\begin{document}
%==========================================
%===========================================
\chapter{引言}
\thispagestyle{empty}
%==========================================================================
\section{论文研究的背景与意义}
海洋工程技术、宇航空间技术和核能科学技术并列作为当代技术革命中三大尖端技术。随着世界经济的飞速发展和人口的不断增加,人类今天正面临着人口、资源和环境三大难题。随着消耗,陆地上的资源正在不断减少,为了生存和发展,人们早已经开始向海洋寻找资源供给。

海洋占地球表面积的71\%,它拥有14亿立方千米的体积。在海底及海洋中,蕴藏着极其丰富的生物资源及矿产资源。仅大洋锰结核的储量就有约3万亿吨,其中锰6233亿吨、铜110亿吨、钴94.5亿吨、镍233亿吨、铁4300亿吨、铝883亿吨等。与陆地上已经探明的储量相比,钴是其5250倍,铁为4.3倍,其它都在33倍以上。海洋还是一个无比巨大的能源库,全世界海洋中储存着约1350亿吨石油,约140万亿立方米的天然气\citep{resource}。因此,大洋海底的探索和开发具有极强的吸引力。

在过去的几十年中,随着人类海洋开发、海洋利用和海洋探索活动的日益增加,人类对水下数据获取和数据传输技术的需求也越来越大。水下的数据由传感器传送到海洋表面,从那里可以将数据经由卫星传发给远处的数据处理中心。

除极低频率外,电磁波在水中的衰减很快,因此不能长距离传播。穿透力较强的长波能力也极为有限,同时还需要体积庞大的天线和大功率发射机来支持,传输速率也非常低\citep{ProakisB2001}。此外,中微子和蓝绿激光也可以用于水下通信\citep{zhongweizi},但是中微子目前还不能方便、可靠的得到,蓝绿激光的传播距离有限\citep{luyiqun1991}。声波可以在水中传播很远的距离,是海洋中的主要信息载体。水下声波是目前最可行的水下探测和通信手段。海洋声学技术在海洋中用于探测、通信、导航和定位等领域,近年来随着海上军事活动及海洋开发的迅速增加,海洋声学技术获得了迅速的发展\citep{liqihu2001,liqihu2002}。

 水声通信是一门综合学科,数字信号处理、无线电通信、移动通信、卫星通信,
 扩频通信以及软件无线电技术和声纳技术的成果都可以借鉴,计算机技术、微电子技术以及高速数字信号处理器(DSP)技术的发展也为水声通信的不断进步提供了坚实的硬件支持。因此,水声通信技术本身可以通过吸收借鉴以及综合各相关学科的优秀成果得到发展。同时水声技术又有着自己的特色,在水声通信发展的初期,由于要求传递的信息量往往不是很大,人们对于水声通信系统的性能要求通常并不是很高。随着人类海洋研究活动的不断深入,对于水声通信系统的效率和可靠性也提出了更高地要求,这使得仅使用传统的通信技术己经不能达到相应的技术要求。因此,许多新方法、新技术引入到了水声通信领域\citep{kuperman1998phase,Stojanovic2005,TXU1997,Stojanovic1996,Catipovic1990,Woodward1996,StojanovicZvonar1996,Stojanovic1994}。

 近年来,在水声通信领域中均衡技术应用得越来越广泛。均衡技术最初应用于无线电通信领域,尤其是电话系统中。之后,随着计算机和网络的发展和普及,在调制解调器中也广泛采用了均衡技术。由于水声通信系统工作于复杂的湖泊、海洋等水声信道之中,因此,在水声通信系统中采用均衡技术消除或减弱信道对信号传输的负面影响是提高系统有效性和可靠性的必要途径。从另一角度看,也正是由于水声通信信道的多样性以及复杂性进一步促进了均衡技术的快速发展。因而,对于水声通信系统中均衡器的应用及其实现的研究已成为近年来水声通信领域的另一研究热点。 
%==========================================================================
\section{水声通信技术发展概况}
人类关于水下发送、接收信息的想法可以追溯到几百年前。早在1490年,意大利的达·芬奇在他的摘记中就有“将长管的一段插入水中,将管的开口放在耳旁就听到远处的航船”的记载。而最具现代意义的真正的水声通信出现在第二次世界大战中,主要用于军事上。例如45年美国开发的用于潜艇间通信的水声电话,采用单边带\footnote{SSB,single sideban
}调制技术。水声电话技术比较成熟,目前仍广泛应用于潜艇间以及潜水员与母船间的通信。但其明显的缺点是功能单一,只能传输语音。

由于客观的需求增加,从二十世纪七十年代开始,军事领域和民用领域都对水声通信技术产生了大量的需求,如在军事领域中,舰艇间的通信,对水下航行器实施监测和导航,水雷的远程声遥控等使得水下通信技术的研究得到人们的高度重视,水声通信技术的重要性也日益突出;在民用领域,渔业资源的开发利用,海上钻井平台和船只的应急维护,水下机器人的研制,水下资源勘探等的发展,对水声通信技术也提出了新的要求。这使得水声通信进入了一个发展相对迅速的阶段。

在水声通信技术快速发展的同时,其它领域的技术,尤其是电信、电子和计算机技术以更为迅猛的速度日新月异地前进。这极大地促进和支持了水声通信技术的发展,水声通信技术发生了深刻地变化,水声通信的面貌焕然一新。

首先发展起来的数字式水声通信系统采用的是非相干水声通信技术。非相干水声通信技术采用多进制频移键控信号(MFSK)加编码的技术克服多径引起的干扰,其带宽利用率较低,传输速率一般为数百bits/s,但其具有好的鲁棒性,因此得到了广泛应用。相干水声通信技术采用多进制相移键控信号(MPSK)、空间分集、自适应均衡器、编码和多普勒补偿等技术,带宽利用率比非相干技术提高了一个数量级,一般传输速率为数千至上万bits/s。相干水声通信技术仍处于迅速发展和完善之中,是当前水声通信技术的发展主流。
由于水声信道具有时变-空变特性,任何单一的通信技术都很难满足水声通信系统高可靠传输数据的要求。因此,在一个水声通信系统中往往是数种通信技术的联合,共同发挥效用,以期达到优良的系统通信性能。在水声通信领域,空间分集技术与自适应均衡技术的结合逐渐成为研究热点。采用这种方式可以同时利用均衡以及分集处理在时间域和空间域获益,从而更为有效地消除水声信道的多径效应。
目前水声通信技术仍在不断发展,各国都在研究把各种新技术应用到水声通信中,如包括采用正交频分复用调制(OFDM)的多载波通信技术、采用码分多址(CDMA)的水下组网技术等\citep{Sozer2000,Rice2000,zhuweiqing1998}。
我国是在八十年代中期开展水声通信技术研究的,中科院声学所、厦门大学、哈尔滨工程大学等都在开展高速水声通信研究,其中中科院声学所在“七五”期间研制了频移键控水声通信样机,在“八五”、“九五”期间开展了相移键控水声通信机的研究\citep{zhuweiqing1998},并被列为国家“863”计划智能机器人主题的预研课题。在“十五”期间声学所承担了某潜水器声学系统的研制任务,在已有技术基础上进一步发展完善,研制一套实用的中程高速水声通信机。厦门大学在厦门湾进行了相干通信的试验\citep{TXU1997}。哈尔滨工程大学在松花湖进行了相干通信和OFDM通信的试验。
%==========================================================================
\section{水声通信中的均衡技术}
水声信道是一多径、色散、时变和深度衰落的信道,声波在其中的传播行为十分复杂。由于水声信道十分恶劣,信道的噪声干扰、时变特性、严重的多径效应和复杂多变的传播环境使水声通信系统中存在着十分严重的码间干扰(ISI)和噪声;界面和介质的起伏也导致了信号的相位变化较剧烈。均衡技术可以补偿信道多径效应和多普勒扩散引起的信道畸变从而减少码间干扰;分集技术则用来补偿信道衰落损耗,而信道编码是通过在发射信息中加入冗余数据来改善数据的纠错性能,从而提高通信系统的接收性能。本文研究了水声相干通信系统的Turbo均衡技术,与信道编码相联合迭代,以达到提高通信系统性能的目的。
\subsection{水声通信中使用均衡器的必要性}
在水声通信系统中,信道的多径效应会产生码间干扰,使发射信号发生畸变,从而在接收端产生误码。

设任一基带信号为:
\begin{eqnarray}
    s(t)=\sum_{n=0}^{\infty}a_ng(t-nT)
    \label{equ:1.1}
\end{eqnarray}
其中,$g(t)$是要选择的基本脉冲形状,用于控制传输信号的频谱特性。$a_n$是由$M$个点组成的信号星座图中选取的传输信息符号的序列,$T$是符号区间($1/T$就是符号率)。

该基带信号经过频率响应为$C(f)$的信道后,接收的信号可以表示为:
\begin{eqnarray}
    r(t)=\sum_{n=0}^{\infty}a_nh(t-nT)+n(t)
    \label{equ:1.2}
\end{eqnarray}
其中,$h(t)=g(t)*c(t)$,$c(t)$是信道冲击响应,$n(t)$代表信中加入的高斯白噪声。为了表示码间干扰,假设接收信号通过一个接收滤波器,然后以每秒$1/T$的采样率进行采样。一般来说,在接收端最佳滤波器是与接收信号脉冲$h(t)$相匹配的匹配滤波器,所以这个滤波器的频率响应为$H^*(f)$。滤波器的输出可以表示为:
\begin{eqnarray}
    y(t)=\sum_{n=0}^{\infty}a_nx(t-nT)+\upsilon(t)
    \label{equ:1.3}
\end{eqnarray}
其中,$x(t)$是接收滤波器的信号脉冲响应,即$X(f)=H(f)H^*(f)$,$\upsilon(t)$是接收滤波器对噪声$n(t)$的响应。如果对$y(t)$在时刻$t=kT$,$k=0,1,2,\cdots$进行采样,则有:
\begin{eqnarray}
    y(kT)=\sum_{n=0}^{\infty}a_nx(kT-nT)+\upsilon(kT)
    \label{equ:1.4}
\end{eqnarray}
记为:
\begin{eqnarray}
    y_k=\sum_{n=0}^{\infty}a_nx_{k-n}+\upsilon_k\;,\quad k=0,1,2,\cdots
    \label{equ:1.5}
\end{eqnarray}
样本值$y_k$可以表示为:
\begin{eqnarray}
    y_k=x_0\left(a_k+\frac{1}{x_0}\sum_{n=0 \atop n\neq
    k}^{\infty}a_nx_{k-n}\right)+\upsilon_k\;,\quad k=0,1,2,\cdots
    \label{equ:1.6}
\end{eqnarray}
$x_0$是任意加权因子,为了方便,将其置为1,那么式(\ref{equ:1.6})可改写为:
\begin{eqnarray}
    y_k=a_k+\sum_{n=0}^{\infty}a_nx_{k-n}+\upsilon_k\;,\quad k=0,1,2,\cdots
    \label{equ:1.7}
\end{eqnarray}
其中,$a_k$代表了在第$k$个采样时刻所期望的符号,而$\sum_{n=0\atop n\neq k}^{\infty}a_nx_{k-n}$项代表了码间干扰。

由(\ref{equ:1.7})可见,水声通信系统为了获得高可靠地数据通信,必须克服由信道产生的码间干扰,而均衡技术是克服信道多径效应、减少码间干扰的有效手段。因此,在水声通信系统中使用均衡器是非常必要的。
\subsection{均衡技术的发展}
均衡技术最早应用于无线电通信领域,主要用于消除由信道响应引起的码间干扰。均衡技术大致分为两大类:线性均衡及非线性均衡器。常用的最大似然序列估计器(MLSE)、判决反馈均衡器(DFE)和Turbo均衡器都属于非线性均衡器。基于最小均方(LMS)算法的线性均衡器算法成熟且简单,但是由于其适用于多种码间干扰不很严重的场合,而对于复杂的水声信道并不适用。非线性均衡器中的最大似然序列估计器由于计算量过大,实际中也难以应用。目前,水声通信系统中应用最广泛的是判决反馈均衡器(DFE)。

由于水声信道是时延和多普勒双扩散信道,上世纪九十年代,在自适应判决反馈均衡器中加入了信号相位补偿器,使均衡器的性能有了长足的进步。

从现代水声通信技术的发展过程来看,自适应均衡技术在其中扮演着重要的角色。利用自适应均衡技术来提高水声通信系统的传输速率和频带利用率,己经成为现代水声通信系统的重要特征。

由于水声信道具有时变-空变及衰落特性,因此,单一的均衡技术已经不能满足现代水声通信系统的要求。如今,自适应均衡技术已经发展到了各种类型的均衡器互相借鉴、融合,以及与通信系统中的其他环节相互联合的阶段。例如,将判决反馈均衡器与最大似然序列估计器相结合\citep{LeNgoc1996}、联合判决反馈均衡器与线性预测方法跟踪衰落信道特性\citep{Blostein1995}、最大似然序列估计器与FIR滤波器联合工作\citep{Pasupathy1995},以及与分集技术相结合的方法\citep{Taylor1995}等等。而另外一种常见的方式则是自适应均衡方法联合各种编码以及调制技术,例如:极大似然序列估计器与编码调制技术相结合\citep{stuber1997}\citep{Younis},判决反馈均衡器结合编码技术以及编码调制技术\citep{Wang1996,Zhou1990,Yang2004},Turbo均衡器与译码器联合迭代\citep{Magniez1999,Raphaeli1997,Vlahoyiannatos2001,Combined2000,Tuchler2002,Tuchler2011,Tuchler2002a,Xiang2003,Yang2007,Yang2005,Anastasopoulos1997,Hanzo2002,Ralf2004}等新方法,新理论。而其中的Turbo均衡技术是最近几年研究的比较广泛的一种均衡技术,该均衡技术能够通过迭代有效的利用译码器反馈的软信息,从而提高均衡性能。下面对这个均衡算法作简要说明。
\subsection{Turbo均衡技术的发展}
Turbo码由法国的C.Berrou\citep{berrou1993}等提出,已经被广泛用于无线和水声通信系统中。Turbo码具有反馈迭代的译码结构和对软信息的利用能力,因此具有极为优越的性能,也能够与其它技术相结合。

利用了Turbo码的译码原理,在接收端,均衡器与译码器的联合处理,即一种迭代算法在对同一组接收到的数据进行重复地均衡和译码,这种处理就是Turbo均衡\citep{douillard1995,Anastasopoulos1997}。
在Turbo均衡算法中,性能最好的是最大后验概率(MAP)算法[],但是其计算复杂度极高,难以在实际应用中使用。因此,人们更倾向于寻找一些降低复杂度而性能次优的算法。Ariyavisitakul和Li\citep{ariyavisitakul1998}提出了一种联合编码和均衡的方案,和以往的接收机不同,这种接收机采用了卷积码和一个判决反馈均衡器(DFE),在DFE中,DFE的前向滤波器中输出的软信息作为维特比译码器的输入,而维特比译码器输出的硬判决信息作为DFE中反馈滤波器的输入,从而形成一个回路,通过不断的迭代来提高均衡和译码的性能。相比于上述算法中DFE利用维特比译码器\citep{Joachim1989}输出的硬判决信息,Marandian等人\citep{Marandian}于2001年提出一种利用译码器软信息的基于DFE的Turbo均衡算法。该算法能够避免硬判决导致的信息损失,从而更有效的提高均衡性能。Wang和Proor\citep{wang1999iterative}提出一种类似Turbo均衡的系统,作为CDMA多用户检测器的一部分。其中基于Turbo均衡的迭代结构采用了一个线性均衡器(LE)和MAP译码器来减少码间干扰。用LE均衡器代替MAP均衡器大大减少了计算量。而Glavieux\citep{glavieux1997turbo}等人提出了用基于线性滤波器的软干扰抵消器(SIC)代替MAP均衡器,以达到减少计算量的目的。这个SIC的系数由基于最小均方误差(LMS)的更新算法给出。后来有人对这种方法进行了改进,为了使LMS算法的结果趋近于MAP均衡的结果,在各种比特信噪比(SNR)下,将MAP均衡的结果进行线性估计,然后存入表中供接收机使用\citep{raphaeli2000}。\citep{wu2000turbo}的近似方法类似于\citep{glavieux1997turbo}中的方法,只是该近似方法适用于已知部分信道响应的磁记录应用中。均衡滤波器的输出被认为是可靠地度量,接收机通过该度量来决定是否使用线性均衡算法代替MAP算法。另外一种降低MAP均衡器复杂度的通用方法是减少状态转移图中的状态个数,具体参考\citep{Berthet2000}。本文提到的这些近似算法都存在一个巨大缺陷\footnote{这个缺陷在经典的Turbo均衡方案中就存在},那就是均衡器的运算复杂度随着信道冲击响应的长度以及符号映射的大小呈指数型增长。T{\"u}chler\citep{Tuchler2002a,Tuchler,Tuchler2011}等人提出基于先验信息MMSE准则的线性Turbo均衡算法。与传统MMSE均衡算法相比,此时的符号分布已经不是独立同分布的了,因此该算法不仅考虑噪声的分布情况同时也考虑了符号的分布。通过利用此中的软信息可以很大程度地提高均衡性能。

目前为止,Turbo均衡性能的好坏,还不能从理论上加以解释,而是从实验中证明其优异的性能。对Turbo均衡算法进行研究,不仅要考虑到其性能的好坏,也要考虑到其算法的复杂度以及计算量的问题。同时也要考虑其在实际工程中能否适用。
%==========================================================================
\section{论文的研究内容及章节安排}
本论文在分析了基于先验信息MMSE准则的线性Turbo均衡算法的基础上,将Turbo编码与该均衡算法相结合,进行理论研究及仿真分析。为了应用于水声通信系统,针对水声信道的特点,提出一种软迭代信道估计算法,并与上述的Turbo均衡方案联合,最终用于水声相干通信中。

论文由七章组成,各章内容安排如下:
\begin{itemize}
  \item \textbf{第一章~引言}\quad
    首先介绍了水声通信的发展概况,接着结合水声信道的特征,说明水声通信系统中使用均衡技术的必要性,以及均衡技术的研究及发展情况。
  \item \textbf{第二章~水声信道特点与自适应均衡技术}\quad
    介绍水声信道的特点以及当前水声相干通信中的自适应均衡技术。
  \item \textbf{第三章~基于先验信息MMSE准则的线性Turbo均衡算法}\quad
      介绍Turbo均衡的系统模型及原理,并与传统均衡方案的比较,且详细介绍基于先验信息MMSE准则的线性Turbo均衡算法并仿真分析其性能。
  \item \textbf{第四章~软迭代信道估计算法}\quad
    介绍了一种针对水声信道特点的软迭代信道估计算法,并仿真分析其性能。
  \item \textbf{第五章~EXIT图在Turbo均衡中的应用}\quad 
      介绍一种外部信息转移图(EXIT),该图能够预测随着迭代次数的增加,自适应Turbo均衡的性能,对于设计Turbo均衡算法有很大的帮助。
  \item \textbf{第六章~水声通信系统Turbo均衡算法湖试数据分析}\quad 
      介绍湖试试验的概括,对采集的数据后处理并对结果进行分析。
  \item \textbf{第七章~结论与展望}\quad 介绍论文的研究结论和未来的研究工作。
\end{itemize}
%%========================================================================
% empty page for two-page print
\ifthenelse{\equal{\ioaside}{T}}{%
  \newpage\mbox{}%
  \thispagestyle{empty}}{}
%==========================================================================
%\end{doucment}
