%\documentclass[twoside,numberorder]{buptthesis}
%\begin{document}
%==========================================
%===========================================
\chapter{引言}
\thispagestyle{empty}
%==========================================================================
\section{选题的背景与意义}
经历了农业时代,工业时代,信息时代,人类发现自己研究的内容始终在陆地上,而神秘且广袤的海洋却涉足不深,因此,有人提出21世纪是海洋的世纪,这说明对海洋的研究已经开始进入人们的视线,而且也越来越深入。

海洋占据地球表面的70\%,蕴涵了丰富的资源,能源,是维持人类社会可持续发展的后勤保障和空间保障,对人类的发展和社会的进步起着至关重要的作用。而且,近些年来,明显感觉到各海洋国家对海洋探测,资源开发的重视,不论是人力财力的投入。

这些都说明对海洋的信息获取和处理的重要性,因此,要构建一个能实时监控海洋的系统。

那么,通信必然成为这个系统非常重要的环节。迄今为止,人类的研究和探测表明,在所有传输载体中,适合海洋通信的只有声波。因为在浑浊、含盐以及各种矿物质比例比较高的海水中,无论我们在陆地通信中常用的光波还是无线电波,它们的传播衰减都是非常大的,因此,在海水中的传输距离非常有限,根本无法满足人类对于海洋活动传播距离的要求,而相比之下,声波却可以在海水中传输的很好,满足要求。
这也是为什么,在海洋通信中,我们一直研究水声通信技术,而不是其他的,通过水声通信,测试仪器可以把探测的数据直接传输到监视端,从而实现实时监测的目的。

水声通信的介质是海水,介于海水的时变性,随机性,复杂性,为了保障通信质量,提高可靠性,我们需要对传输信息进行信道编码,降低误码率。因此,信道编码必是一种研究方向。
%==========================================================================
\section{水声通信技术发展概况}
水声通信是一项在水下收发信息的技术,这个领域涉及到水声学、通信技术、信号处理、电子技术和计算机技术等多种学科。

人类关于水下发送、接收信息的想法可以追溯到几百年前。早在1490年,意大利的达·芬奇在他的摘记中就有“将长管的一段插入水中,将管的开口放在耳旁就听到远处的航船”的记载。而最具现代意义的真正的水声通信出现在第二次世界大战中,主要用于军事上。例如45年美国开发的用于潜艇间通信的水声电话,采用单边带\footnote{SSB,single sideban
}调制技术。现在随着硬件的高速发展,尤其是半导体技术的进步,使得过去很多被认为过于复杂,运算量过于繁重的算法得到逐步的应用;同时,对新的算法发展和研究提供了硬件保障。

水声通信系统的工作原理是这样的:首先将文字、语音、图像等信息,通过电发送机转换成电信号,并由编码器将这些信息进行数字化处理后,换能器又将这些电信号转换成声信号。声信号通过水这一介质,将信息传递到接收换能器。这时声信号又转换为电信号,解码器将数字信息破译后,电接收机再将信息复原为文字、语音、图像。

换能器是一种能将声能和电能相互转换的仪器。有了它,人们就可以在空气中、水中、固体中任意发射和接收不同频率、不同强度的声信号。

最先,水声通信机使用的是模拟信号,可是海洋中的波浪、鱼类、舰船等产生噪声,使海洋中的声场极为混乱,声波在海水中传递时产生“多径干扰信号”这一较大的难题,导致接收到的信号模糊不清。

但是,近代由于数字通信的产生,陆地上的信号干扰被成功解决,水声领域的专家也开始了在该领域进行探索。

因为海水成分很复杂,所以声波传递时被吸收了一部分,而且频率越高吸收就越厉害,对于频率越低的声波,被海水吸收反而越少。专家测得结果,声波频率在4KHZ左右为远距离传递的最佳频率。

水下通信非常困难,主要是由于通道的多径效应、时变效应、可用频带窄、信号衰减严重,特别是在长距离传输中。水下通信相比于有线、无线通信来说速率非常低,因为水下通信采用的是声波而非无线电波。常见的水声通信方式是采用扩频通信技术,如CDMA等。

目前,水声通信技术发展的已经较为成熟,国外很多机构都已研制出水声通信Modem,通信方式主要有:OFDM,扩频以及其它的一些调制方式。此外,现在水声通信技术已发展到网络化的阶段,将无线电中的网络技术(AdHoc)应用到水声通信网络中,可以在海洋里实现全方位、立体化通信(可以与AUV、UUV等无人设备结合使用),但目前只有少数国家实验成功。
%==========================================================================
\section{信道编码的发展概况}
数字信号在传输中往往由于各种原因,使得在传送的数据流中产生误码,从而使接收端产生图象跳跃、不连续、出现马赛克等现象。所以通过信道编码这一环节,对数据码流进行相应的处理,使系统具有一定的纠错能力和抗干扰能力,可极大地避免码流传送中误码的发生。误码的处理技术有纠错、交织、线性内插等。 

 
提高数据传输效率,降低误码率是信道编码的任务。信道编码的本质是增加通信的可靠性。但信道编码会使有用的信息数据传输减少,信道编码的过程是在源数据码流中加插一些码元,从而达到在接收端进行判错和纠错的目的,这就是我们常常说的开销。这就好象我们运送一批玻璃杯一样,为了保证运送途中不出现打烂玻璃杯的情况,我们通常都用一些泡沫或海棉等物将玻璃杯包装起来,这种包装使玻璃杯所占的容积变大,原来一部车能装5000各玻璃杯的,包装后就只能装4000个了,显然包装的代价使运送玻璃杯的有效个数减少了。同样,在带宽固定的信道中,总的传送码率也是固定的,由于信道编码增加了数据量,其结果只能是以降低传送有用信息码率为代价了。将有用比特数除以总比特数就等于编码效率了,不同的编码方式,其编码效率有所不同。

\begin{enumerate}
  \item \textbf{线性分组码}\quad
    线性分组码中的线性是指码组中码元间的约束关系式线性的,而分组则是对编码方法而言,即编码时将每$k$个信息位分为一组进行独立处理,变成长度为$n(n>k)$的二进制码组。

循环码是线性分组码中最主要、最有用的一类,目前对它的研究和应用也很多。它是1957年由Prange首先提出并进行研究的。循环码最引人注目的特点是:首先它可以用线性反馈移位寄存器很容易地实现其编码和伴随式计算,其次由于循环码有很多固有的代数结构,从而可以找到各种简单实用的译码方法。由于循环码具有很好的良好性质,所以它在理论和实践中都是很重要。在循环码中BCH码是其中最主要的一大类。汉明码、R-M码、Golay码、R-S码等均可变换成或者纳入循环码内,1970年发现的Goppa码类中有一子类也属于循环码。
\item \textbf{卷积码}\quad
  卷积码是1955年由Elias等人提出的,是一种非常有前途的编码方法。卷积码将$k$个信息比特编成$n$个比特,但$n$和$k$通常很小,特别适合以串行形式进行传输,时延小。与分组码不同,卷积码编码后的$n$个码元不仅与当前段的$k$个信息有关,还与前面的$N-1$段信息有关,编码过程中相互关联的码元个数为$nN$。卷积码的纠错性能随$N$的增加而增大,而差错率随$N$的增加而指数下降。在编码器复杂性相通的情况下,卷积码的性能优于分组码。
\item \textbf{Turbo码}\quad
  1993年法国人Berrou等在ICC国际会议上提出了一种采用重复迭代(Turbo)译码方法的并行级联码,并采用软输入/输出译码器,可以获得接近Shannon极限的性能,至少在大的交织器和$BER\cong 10^{-5}$条件下,可以达到这种性能。Turbo码的优良性能,受到移动通信领域的广泛重视,特别是在第三代移动通信体制中,非实时的数据通信广泛采用Turbo码。
\end{enumerate}
%==========================================================================
\section{论文的研究内容及章节安排}
本论文主要研究低信噪比下的水声通信的信道编码技术。主要包括卷积码编码和序贯译码算法,RS码编码和BM译码算法,通过matlab仿真确定参数和分析误码率曲线,最后在PC/104上运行。

论文由六章组成,各章内容安排如下:
\begin{itemize}
  \item \textbf{第一章~绪论}\quad
    主要介绍论文研究内容的现实意义已经国内外水声通信技术和信道编码技术发展概况。
  \item \textbf{第二章~信道编码代数基础}\quad
    包括信道编码的基础代数知识,主要包括近世代数里面的群、域、矢量空间、矩阵等。
  \item \textbf{第三章~卷积码的编码与序贯译码算法}\quad
    主要是卷积码序贯译码算法理论基础分析,以及C语言实现流程,译码算法以费诺算法为主要研究对象。
  \item \textbf{第四章~RS码的编码与BM硬判决算法}\quad
    主要是RS码译码算法理论基础分析,以及C语言实现流程,译码算法以BM算法为主要研究对象。
  \item \textbf{第五章~PC/104运行分析}\quad
    主要是把编好的程序在PC/104上运行,观察译码效率,和误码率情况。
  \item \textbf{第六章~结论与展望}\quad 介绍论文的研究结论和未来的研究工作
\end{itemize}
%%========================================================================
% empty page for two-page print
\ifthenelse{\equal{\ioaside}{T}}{%
  \newpage\mbox{}%
  \thispagestyle{empty}}{}
%==========================================================================
%\end{doucment}
