%\begin{document}
\chapter{结论与展望}
\thispagestyle{empty}
%==========================================================================
由于声波可以在水声传播很远的距离,水声通信时水下无线信息传输的主要手段。随着人类在海洋中的活动越来越频繁,需要在水下传世信息的应用需求也越来越多,水声通信技术的应用月越来越广泛。在现在和将来的科学考察、海洋工程建设、海底矿产资源调查与开发以及军事领域等方面,水声通信技术都有着广泛的应用。这些应用也对水声通信的速率和可靠性提出了更高的要求。

在现代水声通信系统中,水声信道面临的主要任务就是客服多径效应产生的码间干扰。而克服多径效应,减少码间干扰的主要手段包括空间分集技术和均衡技术。

本文针对水声信道高速数据传输中的码间干扰抑制问题,研究了Turbo均衡技术以及与其相匹配的软迭代信道估计算法。基于先验信息MMSE准则的线性Turbo均衡技术利用译码器反馈的软信息,以最小化均方误差(MMSE)为准则,估计出发送符号的估计值并通过映射得到关于发送符号的软信息。该软信息作为SISO译码器的输入,通过译码器输出关于发送符号的软信息并反馈给均衡器,从而形成回路。通过迭代次数的增加,均衡器的性能会越来越好。

由于基于先验信息MMSE准则的线性Turbo均衡算法需要信道的信息,为了提高均衡效率,本文针对水声信道的特点,提出一种软迭代信道估计算法,该算法利用软的FOLMS算法估计横向滤波系数以及二阶锁相环估计相位,并通过迭代来提高估计性能。

通过理论分析和计算机仿真,研究了线性Turbo均衡和软迭代信道估计算法对水声信道均衡的性能。此外,还将本文提出的算法进行湖试数据验证,证明了所研究的基于先验信息MMSE准则的线性Turbo均衡算法具有很好的实用价值。
\section{研究工作总结}
本文首先介绍了水声通信技术的发展概况,接着阐述了水声通信系统使用信道均衡技术的必要性及水声均衡技术研究现状。第二章分析了水声信道的特性及其对水声通信的影响。水声信道是多径信道,不同传输距离多径不同,而且信号存在时延扩散;水声信道传输的信号有多普勒频移,而且存在多普勒扩散。因此水声信道是时变的时延-多普勒双扩散信道,并介绍了自适应均衡技术的基本状况以及应用范围,并说明本文研究基于先验信息MMSE准则的线性Turbo均衡算法的原因。

第三章介绍了基于先验信息MMSE准则的线性Turbo均衡算法的基本原理以及简化算法,第四章介绍了由于水声Turbo均衡中的一种软迭代信道估计算法并给出其优越性的验证。湖试数据的处理结果(第五章)验证了本文提出算法的正确性以及可用性。

本文的主要研究工作如下:
\begin{enumerate}
  \item
      提出基于先验信息MMSE准则的线性Turbo均衡算法与T-TCM相结合,并应用于海试、湖试数据处理中。该算法在减少系统运算复杂度的同时,实现均衡器性能的提升。
  \item
      提出一种适用于水声Turbo均衡的软迭代信道估计算法,该算法利用软的FOLMS算法估计横向滤波系数及二阶锁相环估计相位,并通过迭代提高估计性能,从仿真分析以及海试数据处理可以看出,其性能远远高于硬迭代信道估计算法和不带相位估计器的软迭代信道估计算法。
  \item
\end{enumerate}
%==========================================================================
\section{下一步工作展望}
\subsection{比特交织}
TCM\citep{ungerboeck1982,Coded2004,Schlegel2004}把编码和调制作为一个整体考虑,通过最大化编码信号序列间的最小欧氏距离,使其在AWGN信道下取得了非常好的性能。但是,TCM通常体现较低的分集级数,而分集级数是衰落信道下传输的主要设计准则,这使得TCM在Rayleigh衰落信道下性能下降。而比特交织编码方案\citep{zehavi1992,Akay2004,Alvarado,Caire,Sethuraman}通过把传统的二进制纠错码和一组独立的比特交织器连接起来以增加分集级数,是一种带宽有效的编码技术,在复杂度相当的情况下,BICM在衰落信道上的性能要优于TCM,但在AWGN信道下的性能要差。可以使用基于迭代译码的比特交织调制(BICM-ID)。BICM以及其迭代译码(BICM-ID)\citep{LeGoff2006}可以适用用于瑞利信道及高斯信道,性能优于BICM及TCM方案。因此,将Turbo均衡技术与BICM-ID相结合,可以进一步提高系统性能\citep{Antoine2003}。
\subsection{与时间反转技术相结合}
虽然本文提出的算法能够用于水声相干通信系统之中,但是在水声信道特性比较差的情况下,需要均衡器和信道估计器的长度很长,从而会导致运算复杂度的增加,不能实现实时通信,为了改变这一状况,可以采用时间反转技术与本文提出的均衡技术相结合。时间反转技术可以利用多通道信道信息并使得信道的码间干扰大大减少,一般来说均衡器长度为5,信道估计器的长度为4就可以无差错译码。
%%========================================================================
% empty page for two-page print
\ifthenelse{\equal{\ioaside}{T}}{%
  \newpage\mbox{}%
  \thispagestyle{empty}}{}
%%========================================================================
%\end{document}
