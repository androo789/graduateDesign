%致谢
\chapter*{\centerline{致谢}}
\markboth{致谢}{致谢}
\chaptermark{致谢}
\addcontentsline{toc}{chapter}{\xiaosi\heiti 致谢}
\vspace{2em}
本论文的研究工作是在我的导师朱敏研究员的悉心指导下完成的。朱老师做事严谨、知识丰富且待人和善。作为科研工作者,朱老师有渊博的知识和丰富的经验;作为项目负责人,大到总体框架,小到一个小的电路板的设计都了然于胸;作为一名导师,更是谆谆善诱、诲人不倦。正是在这样的导师的指导下,我才能在技术上和做事上有了较大的提高。在此,对朱老师衷心的说一声“谢谢”!

感谢朱维庆教授。朱维庆教授是实验室的创始人,正是有了老先生的指导,实验室多年来蒸蒸日上,为我们研究生的成长提供了优越的环境。当日选择来中科院读研究生,也是希望能目睹中科院老前辈们的工作风采,以受熏陶,如今愿望成真,老先生的工作热情鼓舞了我追求自己梦想的勇气。

感谢武岩波老师。初到实验室,不知所措,是他,让我慢慢适应环境,他的细心指点、自己的学习体会、学习规划以及未来研究方向,都热情地给我讲述,让我感慨自己知识的狭窄,惊喜于未来知识的广阔。在毕设开始以致结束整个过程,武岩波老师总是为我解开各种疑惑和问题,并对设计提出非常有意义的建议,在论文撰写的时候,他也给我提出很多结构上的问题,让论文条理清晰。再次感谢武岩波老师。

感谢水声通信网项目组的徐立军、傅翔、李欣国、魏振坤、孙兴涛、邢泽平等,感谢他们对我学习和生活上的指导和照顾,感谢他们陪伴我度过千岛湖实验那段美好的时光。

感谢王季煜师兄和李海莲师姐。他们对学业认真的态度,对问题冷静的分析方法,都是值得我学习的。他们不仅在学习上给予我诸多指点,在生活上也对我颇多照顾。

感谢我的同学崔兴隆、赵二亮、马驰、陈若婷,感谢我的师弟师妹许浩、张威、李丹丹、曹松军、樊艳强,陪伴我度过研究生时光。

最后,我要感谢的是我最亲爱的父母。在我二十多年的成长过程中,你们无时无刻无私地关怀和奉献,是我独在异乡求学的最大精神支柱,也是我可以依偎的最温馨港湾。你们是我永远的牵挂和眷念!谨以此文献给我挚爱的双亲。
\clearpage{\pagestyle{empty}\cleardoublepage}

