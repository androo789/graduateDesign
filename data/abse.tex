
%% 英文摘要
\chapter*{\bfseries\centerline{ABSTRACT}}
\pagestyle{plainabstracte}
\chaptermark{ABSTRACT}
\addcontentsline{toc}{chapter}{\xiaosi\heiti ABSTRACT} 
\thispagestyle{plainabstracte}
\vspace{1em}

In the low signal to noise ratio of acoustic communication system,
   convolutional code and RS code are adopted to improve the reliability of
   information transmission.For convolutional code, using sequential
   decoding method to simplify the decoding algorithm and to improve the
   efficiency. The fano algorithm is one of the sequential decoding methods
   which is small space spending, relatively simple to achieve and used
   widely in reality. So, finally taking it as the decoding algorithm. For RS code, BM algorithm is almost the most widely one among all decding algorithms.

Determining the coding parameters by matlab simulation and test decoding
efficiency in PC/104.

Through test data, several conclusions can be obtained. With sequential
decoding algorithm, the error of convolutional code which constraint length is
24 can be very good to meet the requirements and be consistent with the
theory of the book, also the running time is with delay. Therefore, the
system needs to a buffer to store the decoding data; The $RS(15,9,7)$with BM decoding algorithm, the error rate has been greatly improved and the running time is very short. So the system does not need to buffer the decoding data.

In the low signal to noise ratio of acoustic communication system,
   convolutional code and RS code are adopted to improve the reliability of
   information transmission.For convolutional code, using sequential
   decoding method to simplify the decoding algorithm and to improve the
   efficiency. The fano algorithm is one of the sequential decoding methods
   which is small space spending, relatively simple to achieve and used
   widely in reality. So, finally taking it as the decoding algorithm. For RS code, BM algorithm is almost the most widely one among all decding algorithms.

Determining the coding parameters by matlab simulation and test decoding
efficiency in PC/104.

Through test data, several conclusions can be obtained. With sequential
decoding algorithm, the error of convolutional code which constraint length is
24 can be very good to meet the requirements and be consistent with the
theory of the book, also the running time is with delay. Therefore, the
system needs to a buffer to store the decoding data; The $RS(15,9,7)$with BM decoding algorithm, the error rate has been greatly improved and the running time is very short. So the system does not need to buffer the decoding data.
\vspace{1em}

\thispagestyle{plainabstracte}
\noindent\textbf{Keywords}

Low SNR \quad Underwater acoustic communication \quad Convolutional code
\quad RS code
