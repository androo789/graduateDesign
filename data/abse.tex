
%% 英文摘要
\chapter*{\bfseries\centerline{ABSTRACT}}
\pagestyle{plainabstracte}
\chaptermark{ABSTRACT}
\addcontentsline{toc}{chapter}{\xiaosi\heiti ABSTRACT} 
\thispagestyle{plainabstracte}
\vspace{1em}

As sound waves can be spread very far away in underwater acoustic condition, it
becomes the principal means of underwater wireless information transmission. As
human activities become more frequent in the ocean, the requirement to transmit
underwater information is also more and more, the application of underwater
acoustic communication technology is becoming more and more widely. Now and in
the future, in the field of scientific research, Marine construction, ocean
mineral resources research\&development and military fields, etc., underwater acoustic communication technology has a wide range of applications. These applications also require higher rate and more reliability of the underwater acoustic communication.

In the modern underwater acoustic communication system, to overcome intersymbol interference produced by multipath effect is the main task for underwater acoustic channel. While the main methods to overcome the multipath effect and reduce intersymbol interference include space diversity technology and equalization technology.

Based on high speed data transmission of intersymbol interference suppression problem in underwater acoustic channel, turbo equalization technique and the soft iterative channel estimation algorithm is studied in this paper. The linear turbo equalization of minimum mean squared error(MMSE) equalization using a priori information(MMSE-LE) can make use of soft information feedback by decoder, estimate transmission symbols and map the soft information about transmission symbols. SISO decoder takes this soft information as input and produces soft extrinsic information for equalization. The equalizer's performance will be improved as iterating.

Due to MMSE-LE requires channel information, to improve equalizer's performance, this paper proposed soft iterative channel estimation algorithm according to the characteristics of underwater acoustic channel. This algorithm applies soft fast self-optimized LMS to estimate transverse filter coefficients and second order phase-locked loop to estimate phase while through iterating to improve estimated performance.

Through theoretical analysis and computer simulation, this paper researches the performance of MMSE-LE and soft iterative channel estimation algorithm. In addition, the proposed algorithms are applied to process data obtained by ocean and lake experiments, and the processing result verifies that MMSE-LE and soft iterative channel estimation algorithm has the very good practical value.

In this paper, the main research work is as follows:
\begin{enumerate}
  \item
      Propose a algorithm about combining MMSE-LE with T-TCM, and apply the
      algorithm to process data obtained by Ocean\&Lake experiment. This algorithm not only reduces system computation complexity, but also realizes the equalizer performance boost.
  \item
      Propose a soft iterative channel estimation algorithm that is suitable for underwater acoustic turbo equalization. This algorithm applies soft fast self-optimized LMS to estimate transverse filter coefficients and second order phase-locked loop to estimate phase while through iterating to improve estimated performance. From the simulation analysis and experiment data processing result, its performance is much better than the hard iterative channel estimation algorithm and soft iterative channel estimation algorithm without phase estimator. 
\end{enumerate}

\vspace{1em}

\thispagestyle{plainabstracte}
\noindent\textbf{Keywords}

Underwater Acoustic Coherent Communication \quad Underwater Acoustic
Channel \quad Fast Self-Optimized LMS(FOLMS) \quad Adaptive Equalization
\quad Turbo Equalization \quad Second-order phase-locked loop \quad Bit
Interleave \quad Time Reverse
